I think criteria including 
1) highly resolved and realistic BEV charging profile modeling + endogenous capacity expansion 
2) modeling of controlled charging and V2G 
are the two most important features of our study. Can we downsize to these there columns instead?

Matt's reviewed literature does not necessitate charging intervention. Studies I have reviewed all have charging intervention as an integral part. I included studies from the lit review in his 2024 EST paper that are relevant to our study.

There are three types of studies on charging intervention that I have reviewed so far: 
1) marginal emission factors: Holland 2022 and Gagnon 2022, maybe tu et al. 2019 or Gai et al. 2019 (which are essentially the same)
2) operational cost model based short-term studies: studies on PEV charging intervention that I have reviewed that do not have capacity expansion and V2G
3) capacity expansion model based long-term studies on PEV charging intervention . 

Owens et al. 2022 and Weis et al. 2014 are the closest to ours. Owens has two drawbacks: 1) aggregated PEV charging profile, due to reliance on high-level capacity expansion model. Weis et al. 2014 does not consider V2G, use weeks of operation to represent yearly operation.

The three column arrangement should reflect this well.



%%% New lit review table
\begin{table*}[!ht]
\caption{Peer-reviewed publications estimating power system emissions consequences of increasing PEV charging load}\label{tab:method}
\begin{center}
\centering
\centering
\begin{tblr}{
  cell{2-20}{2-4} = {c},
  hline{1-2,19-20} = {-}{},
}
                      & {Endogenous power\\system operation\&\\capacity expansion \\} & {Highly-resolved\\and realistic\\ PEV charging\\ profile modeling}&{Controlled\\ charging}&{V2G} \\
Weis et al. 2014      
\cite{weis_estimating_2014}     & $\checkmark$  &               & $\checkmark$  &               \\
Carrión et al. 2015   
\cite{carrion_operation_2015}   & $\checkmark$  &               & $\checkmark$  &               \\
Weis et al. 2015
\cite{weis_emissions_2015}      &               &               & $\checkmark$  &               \\
Forrest et al. 2016
\cite{forrest_charging_2016}    &               & $\checkmark$  & $\checkmark$  &               \\
Tarroja et al. 2016
\cite{tarroja_assessing_2016}   &               & $\checkmark$  & $\checkmark$  &               \\
Nunes  Brito 2017
\cite{nunes_displacing_2017}    &               & $\checkmark$  & $\checkmark$  &               \\
Nopmongcol et al. 2017
\cite{nopmongcol_air_2017}      & $\checkmark$  &               &               &               \\
Brown et al. 2018
\cite{brown_synergies_2018}     & $\checkmark$  &               & $\checkmark$  & $\checkmark$  \\
Manríquez et al. 2020
\cite{manriquez_impact_2020}    & $\checkmark$  &               & $\checkmark$  & $\checkmark$  \\
Shi et al. 2020
\cite{shi_integration_2020}     &               &               & $\checkmark$  & $\checkmark$  \\
Tu et al. 2020
\cite{tu_electric_2020}         &               &               & $\checkmark$  &               \\
Sheppard et al. 2021
\cite{sheppard_private_2021}    & $\checkmark$  & $\checkmark$  &               &               \\
Gagnon \& Cole 2022
\cite{gagnon_planning_2022}     & $\checkmark$  &               &               &               \\
Holland et al. 2022
\cite{holland_why_2022}         &               &               &               &               \\
Owens et al. 2022
\cite{owens_can_2022}           & $\checkmark$  &               & $\checkmark$  & $\checkmark$  \\
Jenn et al. 2023
\cite{jenn_emissions_2023}      &               & $\checkmark$  & $\checkmark$  &               \\
Bruchon et al. 2024
\cite{bruchon_cleaning_2024}    &               & $\checkmark$  &               &               \\
This study                      & $\checkmark$  & $\checkmark$  & $\checkmark$  & $\checkmark$  \\
\end{tblr}
\end{center}
\end{table*}


