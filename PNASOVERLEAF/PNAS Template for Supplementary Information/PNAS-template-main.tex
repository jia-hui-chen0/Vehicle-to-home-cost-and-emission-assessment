\documentclass[9pt,twoside,lineno]{pnas-new}
% Use the lineno option to display guide line numbers if required.
\usepackage{multirow}
\usepackage{amsmath,mathtools,longtable}
\templatetype{pnassupportinginfo}

\title{Your main manuscript title}
\author{Author1, Author2 and Author3 (complete author list)}
\correspondingauthor{Corresponding Author name.\\E-mail: author.two@email.com}

\begin{document}

%% Comment out or remove this line before generating final copy for submission; this will also remove the warning re: "Consecutive odd pages found".
\instructionspage  

\maketitle

%% Adds the main heading for the SI text. Comment out this line if you do not have any supporting information text.
\SItext

\section*{Supplementary Literature Review}

This section provides supplementary literature review to the main body (\ref{tab:method}). Current studies assess emission impacts of plug-in electric vehicle adoption and charging interventions mainly with two approaches, focusing either on the power system operation or on the power system capacity expansion. The first approach assesses emission impacts induced by grid intervention with a fixed power system\cite{holland_why_2022,tu_electric_2020,chen_emission_2022,wang_value_2021, shi_integration_2020,ryan_comparative_2016}. Tu et al. 2020 and Holland et al. 2022 used marginal emission factors to evaluate emission impacts of PEV charging intervention.\cite{tu_electric_2020,holland_why_2022} Chen et al. 2018 used a power system dispatch model to assess the emission impacts of charging intervention in fixed power systems \cite{chen_impacts_2018}. However, Owens et al. and Gagnon et al. \cite{owens_can_2022,gagnon_planning_2022} note that, in the long run, grid interventions like EV adoption can induce changes in the trajectory of power system generator capacity expansion, and this structural change of the power system also has climate implications.%\cite{,gagnon_planning_2022}

Other relevant studies employ capacity expansion models (CEMs). However, CEMs typically do not model granular short-term power system constraints, and many of these studies make simplification assumptions about grid operation \cite{brown_synergies_2018, owens_can_2022, manriquez_impact_2020}.  Brown et al. 2018 assessed cost reduction effects of grid flexibility options provided by the building sector and the transportation sector \cite{brown_synergies_2018}. Though their model included PEV with smart charging and V2G, it represented PEV flexibility with stationary storage of 50\% capacity of PEVs. This simplification can be necessary for CEMs but lacks representation of operational constraints that are vital to understanding PEV flexibility. Weis et al. 2014 managed to join the force of an operational cost model and a capacity expansion model\cite{weis_estimating_2014}. The study ran power system simulation for 4 representative weeks of a year and used the results as inputs for a capacity expansion model. As a result, the study's scalability was limited. Moreover, the study did not model V2G. Owens et al. 2022, based on a CEM, represented V2G operation in the model with detailed operational constraints, informed by real-world travel records \cite{owens_can_2022}. However, they only allow V2G to participate in the ancillary service market but not in the energy market, which limits the potential of V2G. Moreover, the study only considers a hypothetical optimal power system that is built from scratch and does not consider effects of installed fossil fuel generators on grid interventions, which, can increase system emissions with large scale deployment of energy storage and\cite{hittinger_bulk_2015} While the investigation done by existing studies have been insightful and fruitful, a holistic view that combines power system operation and capacity expansion helps answer the question better, as it is the short-term operational constraints that ultimately determines the system operation outcome. 

Furthermore, by ignoring transmission constraints between grid regions, along with the possibility that generation and EV loads are added in different regions, the study omits probable spatial mismatch between V2G and generation. As for the spatial nuances, V2G service is best available in load centers \cite{muratori_impact_2018,kahn_green_2009}, whereas much of variable renewable energy generation capacity is located in distant and rural areas \cite{brown_spatial_2020,vanatta_costs_2022}. This mismatch of the spatial distribution of demand and supply also affects the value of grid flexibility options such as flexible PEV load.

%%% New lit review table
\begin{table*}[!ht]
\caption{Comparison of method and materials of existing studies and this study}\label{tab:method}
\begin{center}
\begin{tabular}{@{}llllllll@{}}
\toprule
{\begin{tabular}[c]{@{}l@{}}Author and\\ year\end{tabular}} & \begin{tabular}[c]{@{}l@{}}Modeling of \\operational constraints\\ of grid intervention\end{tabular}
 & \begin{tabular}[c]{@{}l@{}}Realistic BEV\\ charging\\ profiles \end{tabular}
 &\begin{tabular}[c]{@{}l@{}}Endogenous\\capacity\\expansion\end{tabular}
 &  \begin{tabular}[c]{@{}l@{}}Consideration of\\existing fossil\\fuel generators \end{tabular}
  &\begin{tabular}[c]{@{}l@{}}Analysis of\\ impacts of grid\\  intervention on\\  VRE deployment   \end{tabular}
   &\begin{tabular}[c]{@{}l@{}}Endogenous\\power system\\operation modeling\end{tabular}
   &\begin{tabular}[c]{@{}l@{}}V2G\\modeling\end{tabular}\\
 \midrule
\begin{tabular}[c]{@{}l@{}}Owens et al.\\ 2022 \cite{owens_can_2022}\end{tabular} & \multicolumn{1}{c}{$\checkmark$} & & \multicolumn{1}{c}{$\checkmark$} & &  \multicolumn{1}{c}{$\checkmark$} & \multicolumn{1}{c}{$\checkmark$} & \multicolumn{1}{c}{$\checkmark$}\\
\begin{tabular}[c]{@{}l@{}}Holland et al.\\ 2022 \cite{holland_why_2022}\end{tabular} & &   &  & \multicolumn{1}{c}{$\checkmark$} &  & \\
\begin{tabular}[c]{@{}l@{}}Gagnon \& Cole\\2022 \cite{gagnon_planning_2022}\end{tabular} & & &\multicolumn{1}{c}{$\checkmark$}  &\multicolumn{1}{c}{$\checkmark$}&  \multicolumn{1}{c}{$\checkmark$}& \multicolumn{1}{c}{$\checkmark$}& \\
\begin{tabular}[c]{@{}l@{}}Carrión et al.\\ 2015 \cite{carrion_operation_2015}\end{tabular} & \multicolumn{1}{c}{$\checkmark$} &  & \multicolumn{1}{c}{$\checkmark$} & \multicolumn{1}{c}{$\checkmark$}& \multicolumn{1}{c}{$\checkmark$} & & \\
\begin{tabular}[c]{@{}l@{}}Shi et al.\\ 2020 \cite{shi_integration_2020}\end{tabular} & \multicolumn{1}{c}{$\checkmark$}&  & & \multicolumn{1}{c}{$\checkmark$} & & \multicolumn{1}{c}{$\checkmark$} & \multicolumn{1}{c}{$\checkmark$} \\
\begin{tabular}[c]{@{}l@{}}Manríquez et al.\\ 2020 \cite{manriquez_impact_2020}\end{tabular} & \multicolumn{1}{c}{$\checkmark$} &  &  & \multicolumn{1}{c}{$\checkmark$} &  & \multicolumn{1}{c}{$\checkmark$} & \multicolumn{1}{c}{$\checkmark$} \\
\begin{tabular}[c]{@{}l@{}}Jenn et al.\\ 2020 \cite{jenn_environmental_2020}\end{tabular} & &  \multicolumn{1}{c}{$\checkmark$} &  \multicolumn{1}{c}{$\checkmark$} & \multicolumn{1}{c}{$\checkmark$} &  & \multicolumn{1}{c}{$\checkmark$} &  \\
\begin{tabular}[c]{@{}l@{}}Weis et al.\\ 2014 \cite{weis_estimating_2014}\end{tabular} & \multicolumn{1}{c}{$\checkmark$} & \multicolumn{1}{c}{$\checkmark$} & \multicolumn{1}{c}{$\checkmark$} & & \multicolumn{1}{c}{$\checkmark$} & \multicolumn{1}{c}{$\checkmark$} & \\
\begin{tabular}[c]{@{}l@{}}Weis et al.\\ 2015 \cite{weis_emissions_2015}\end{tabular} &\multicolumn{1}{c}{$\checkmark$} & \multicolumn{1}{c}{$\checkmark$} & & \multicolumn{1}{c}{$\checkmark$} & \multicolumn{1}{c}{$\checkmark$} & \multicolumn{1}{c}{$\checkmark$} &  \\
\begin{tabular}[c]{@{}l@{}}Nunes \& Brito\\ 2017 \cite{nunes_displacing_2017}\end{tabular}& \multicolumn{1}{c}{$\checkmark$} & \multicolumn{1}{c}{$\checkmark$} &  &  &  & \multicolumn{1}{c}{$\checkmark$} &\multicolumn{1}{c}{$\checkmark$} \\ 
\begin{tabular}[c]{@{}l@{}}Forrest et al.\\ 2016 \cite{forrest_charging_2016}\end{tabular} & \multicolumn{1}{c}{$\checkmark$} & \multicolumn{1}{c}{$\checkmark$} &   &\multicolumn{1}{c}{$\checkmark$} & & \multicolumn{1}{c}{$\checkmark$} & \multicolumn{1}{c}{$\checkmark$} \\
\begin{tabular}[c]{@{}l@{}}Tarroja et al.\\ 2016 \cite{tarroja_assessing_2016}\end{tabular} & \multicolumn{1}{c}{$\checkmark$} & \multicolumn{1}{c}{$\checkmark$} &   & & & \multicolumn{1}{c}{$\checkmark$}& \multicolumn{1}{c}{$\checkmark$}\\
\begin{tabular}[c]{@{}l@{}}This study \end{tabular}& \multicolumn{1}{c}{$\checkmark$} & \multicolumn{1}{c}{$\checkmark$} & \multicolumn{1}{c}{$\checkmark$} & \multicolumn{1}{c}{$\checkmark$} & \multicolumn{1}{c}{$\checkmark$} &\multicolumn{1}{c}{$\checkmark$}&\multicolumn{1}{c}{$\checkmark$} \\
\bottomrule
\end{tabular}
\end{center}
\end{table*}

\section*{Materials}
\subsection{Variable renewable energy capacity installation scenarios}

The corresponding relationship between variable renewable energy capacity installation capacity and VRE penetration is shown in Figure \ref{fig:capacityexp}.

\begin{figure}[!ht]
    \centering
    \includegraphics[width=\textwidth]{capacity_journal.pdf}
    \caption{VRE generation capacity installation across VRE penetration levels.}
    \label{fig:capacityexp}
\end{figure}

\section*{Methods}
\subsection{Variable renewable energy cost estimation}

We use cost modeling results from National Renewable Energy Laboratory's Annual Technology Baseline (ABT) database to assess fixed costs of solar and wind generators \cite{vimmerstedt_2022_2022}. 2035 Conservative, Moderate, and Advanced scenarios in ABT correspond to Conservative, Best-guess, and Advanced scenarios. Annualized capital costs are calculated with an interest rate of 5\% and a capital recovery term of 30 years. The costs of three scenarios are shown in Table \ref{tab:costs}.

\begin{table}[H]\label{fig:mix}
    \caption{Capital expenditures and fixed O\&M costs of VRE sources, under different cost scenarios \cite{vimmerstedt_2022_2022}}.
    \centering
    \begin{tabular}{ccccc}
    \hline
    Technology &  & Advanced & Best-guess & Conservative \\ \hline
    \multirow{2}{*}{On-shore wind} & Capital expenditures (USD/kW) & 1034 & 1093 & 1231 \\
     & Annual O\&M (USD/kW) & 21.6 & 26.1 & 28.1 \\
    \multirow{2}{*}{Off-shore wind} & Capital expenditures (USD/kW) & 2397 & 2709 & 3414 \\
     & Annual O\&M (USD/kW) & 75.1 & 84.5 & 104.8 \\
    \multirow{2}{*}{Solar} & Capital expenditures (USD/kW) & 632 & 829 & 1101 \\
     & Annual O\&M (USD/kW) & 13.1 & 15.5 & 18.9 \\ \cline{1-5} 
    \end{tabular}
    \label{tab:costs}
\end{table}

\subsection{PJM System Topography}

The original modeling of the PJM Interconnection as 5 transmission constrained regions (TCR) was developed in Lueken et al. 2014 and Weis et al. 2014 \cite{lueken_effects_2014,weis_estimating_2014}, which was later updated by Bruchon et al. 2024 \cite{bruchon_cleaning_2024}. The TCRs and transmission interfaces connecting TCRs are shown in Figure \ref{fig:TCR}. Transmission lines are aggregated into 5 inter-regional transmission interfaces, shown in Figure \ref{fig:trans}. Transmission capacity of each interfaces are extracted from PJM website (PJM Data Miner. 

\begin{figure}[!ht]
    \centering
    \includegraphics{Figures/TCR.png}
    \caption{Map of Transmission Constrained Regions (TCRs) and Transmission Interconnections (TIs).}
    \label{fig:TCR}
\end{figure}

\begin{figure}[!ht]
    \centering
    \includegraphics{Figures/Transmission.png}
    \caption{Map of PJM 500kV transmission lines (white lines) and transmission interfaces (red lines).  Most interfaces contain multiple 500kV lines. The map illustrates model representation of PJM in Bruchon et al. 2024.}
    \label{fig:trans}
\end{figure}

\subsection{Power system model formulation}

\begin{table}[!ht]
  \begin{center}
    \caption{Nomenclature: sets, decision variables, and input parameters}
    \label{tab:notation}
   \scriptsize
    \begin{tabular}{lll}
      Label & Type & Description \\
      \hline
      $\mathcal{I}$ & Set & Generating units\\
      $\mathcal{K}$ & Set & Storage units \\
      $\mathcal{R}$ & Set & Transmission constraint regions (TCRs)\\
      $\mathcal{T}$ & Set & Timesteps\\
      $\mathcal{V}$ & Set & Plug-in electric vehicle schedule groups\\
      \hline
     
      $c^{\mathrm{VEH}}_{v,r,t}$ & Variable & Charge level (MWh) of vehicle schedule group $v$ in TCR $r$ at time $t$\\
      $c^{\mathrm{STR}}_{k,t}$ & Variable & Charge level (MWh) of storage unit $k$ at time $t$\\
      $p^{\mathrm{GEN}}_{i,t}$ & Variable & Power generated by unit $i$ at time $t$ \\
      $p^{\mathrm{SGEN}}_{r,t}$ & Variable & Power generated by solar PV generators in TCR $r$ at time $t$ \\
      $p^{\mathrm{WGEN}}_{r,t}$ & Variable & Power generated by wind generators in TCR $r$ at time $t$ \\
      $p^{\mathrm{STRD}}_{k,t}$ & Variable & Power discharged $(+)$ by storage unit $k$ at time $t$ \\
      $p^{\mathrm{STRC}}_{k,t}$ & Variable & Power charged $(+)$ by storage unit $k$ at time $t$ \\ 
      $p^{\mathrm{RR}}_{r,r',t}$ & Variable & Power imported $(+)$ from TCR $r'$ to TCR $r (r\neq r')$ at time $t$\\
      $p^{\mathrm{VR}}_{v,r,t}$ & Variable & Power charged $(+)$ by vehicle schedule group $v$ in TCR $
r$ at time $t$\\
      $p^{\mathrm{DR}}_{v,r,t}$ & Variable & Power discharged $(+)$ by vehicle schedule group $v$ in TCR $r$ at time $t$\\
      % reserve
      $a^{\mathrm{GEN}}_{i,t}$ & Variable & Reserve provided by generator unit $i$ at time $t$ \\
      $a^{\mathrm{STR}}_{k,t}$ & Variable & Reserve provided by storage unit $k$ at time $t$ \\
      $a^{\mathrm{VR}}_{v,r,t}$ & Variable & Reserve provided by vehicle schedule group $v$ in TCR $r$ at time $t$\\
      $a^{\mathrm{GENSP}}_{i,t}$ & Variable & Spinning reserve provided by generator unit $i$ at time $t$ \\
      $a^{\mathrm{
STRSP}}_{k,t}$ & Variable & Spinning reserve provided by storage unit $k$ at time $t$ \\
      $a^{\mathrm{VRSP}}_{v,r,t}$ & Variable & Spinning reserve provided by vehicle schedule group $v$ in TCR $
r$ at time $t$\\
      $s^{\mathrm{STARTUP}}_{i,t}$ & Variable & Slack variable for startup cost \\
      $u_{i,t} $ & Variable & Binary variable that is equal to 1 if unit $i$ is online in period $t$ and 0 otherwise \\
      \hline
      $A_{v,t}$ & Parameter & Availability of vehicle schedule group $v$ to charge at time $t$ (\%) \\
      $B_{v}$ & Parameter & Battery capacity (MWh) of vehicle grouping $v$\\
      ${c^{\mathrm{STRMAX}}_{k}} $ & Parameter & Max charge level (MWh) of storage unit $k$ \\
      ${c^{\mathrm{STRMIN}}_{k}} $ & Parameter & Min charge level (MWh) of storage unit $k$ \\
      $C^{\mathrm{VARGEN}}_i$ & Parameter & Variable cost of generating unit $i$ \\
      $C^{\mathrm{STARTGEN}}_i$ & Parameter & Startup cost of generating unit $i$ \\
      $p^{\mathrm{SGENMAX}}_{r,t}$ & Parameter & Maximum power generated by solar PV generators in TCR $r$ at time $t$ \\
      $p^{\mathrm{WGENMAX}}_{r,t}$ & Parameter & Maximum power generated by wind generators in TCR $r$ at time $t$ \\
      $P_{r,t} $ & Parameter & Power demand in TCR $r$ at time $t$ \\
      $p^{\mathrm{RRMAX}}_{r,r',t} $ & Parameter & Max power flow on interface $(r,r')$ at time $t$ \\
      $M_{v,t}$ & Parameter & Miles traveled per car in vehicle schedule group $v$ at time $t$ \\
      $N_{v,r}$ & Parameter & Number of vehicles in schedule group $v$ in TCR $r$ \\
      ${p^{\mathrm{GENMIN}}_{i}} $ & Parameter & Min generation from unit $i$ at time $t$ \\
      ${p^{\mathrm{GENMAX}}_{i}} $ & Parameter & Max generation from unit $i$ at time $t$ \\
      $p^{\mathrm{STRMAX}}_{k} $ & Parameter & Max discharge rate from storage unit $k$ \\
      $R_{i} $ & Parameter & Ramp rate limit of unit $i$ \\
      $R^{\mathrm{MAXCHG}}_v$ & Parameter & Max charge rate (MW) of vehicle schedule group $v$\\
      $T_{i}^{\mathrm{UPMIN}} $ & Parameter & Minimum uptime of unit $i$ \\
      $T_{i}^{\mathrm{DTMIN}} $ & Parameter & Minimum downtime of unit $i$ \\
      $C^{\mathrm{GENRES}}_{i}$ & Parameter & Reserve provision cost of unit $i$\\
      $T^{\mathrm{UPSTART}}_{i} $ & Parameter & Number of timesteps unit $i$ has been online at initial timestep $t=1$ \\
      $T^{\mathrm{DTSTART}}_{i} $ & Parameter & Number of timesteps unit $i$ has been offline at initial timestep $t=1$ \\
      $T_v^{\mathrm{DEPART}}$ & Parameter & Timestep at which vehicle schedule group $v$ departs home\\      
      $\eta^{\mathrm{CHG}}_{v}$ & Parameter & Charging and discharging efficiency (\%) of vehicle schedule group $v$\\
      $\eta^{\mathrm{STR}}_k$ & Parameter & Efficiency (\%) of storage unit $k$\\
      $\eta^{\mathrm{VEH}}_{v}$ & Parameter & Driving efficiency (MWh/mile) of vehicle schedule group $v$\\
      $\eta^{\mathrm{TR}}$ & Parameter & Transmission efficiency (\%)\\
      $p^{\mathrm{S}}_r$ & Parameter & Solar PV capacity of region $r$ (MW)\\
      $\phi^{REG}$ & Parameter & Ratio of load regulation reserve requirement to load (\%)\\
      $\phi^{WIND}$ & Parameter & Ratio of wind regulation reserve requirement to regional wind generation (\%)\\     
      $\phi^{SOL}$ & Parameter & Ratio of solar PV regulation reserve requirement to regional solar PV installed capacity (\%)\\     
      $\phi^{SPIN}$ & Parameter & Ratio of spinning reserve requirement to load (\%)\\
      \hline

    \end{tabular}
  \end{center}
\end{table}


We in this study updated and adapted the unit commitment economic dispatch model formulation of Weis. et al. 2016 and Bruchon et al. 2024 to enable V2G functionalities \cite{weis_consequential_2016,bruchon_cleaning_2024}. The mixed integer optimization problem is formulated as follows in Table \ref{tb:eqns} and Table \ref{tb:eqns2}:\\
\newpage

\begin{table}
\centering
  \caption{Optimization problem formulation (continued).}
  \label{fig:eqns2}
   \scriptsize

\allowdisplaybreaks
\begin{align}
\multicolumn{5}{l}{
\begin{multlined} \textrm{Minimize } Z =
\sum_{t\in \mathcal{T}}{\sum_{i\in I_r} {\left({C}^{\mathrm{VARGEN}}_i  p^{\mathrm{GEN}}_{i,t} + s^{\mathrm{STARTUP}}_{i,t}+C^{\mathrm{GENRES}}_{i} a^{\mathrm{GEN}}_{i,t} \right)}}\\
+\sum_{t\in \mathcal{T}}{\sum_{k \in \mathcal{K}}{C^{\mathrm{STR}}_{k} a^{\mathrm{STR}}_{k,t}}}+\sum_{t\in \mathcal{T}}{\sum_{v \in \mathcal{V}}{\sum_{r \in \mathcal{R}}}{C^{\mathrm{VEH}} a^{\mathrm{VEH}}_{v,r,t}}} 
\end{multlined}
} \nonumber \\ \text{Subject to}\nonumber \\
\multicolumn{4}{l}{System constraints:} \nonumber \\
&
\begin{multlined}
  P_{r,t}=\sum_{i\in \mathcal{I}_r}p^{\mathrm{GEN}}_{i,t}+\sum_{k\in \mathcal{K}_r}p^{\mathrm{STRD}}_{i,t} -\sum_{k\in \mathcal{K}_r}p^{\mathrm{STRC}}_{i,t} -\sum_{r'\in\mathcal{R}\neq r}(p^{\mathrm{RR}}_{r',r,t})\\
  +\sum_{r'\in\mathcal{R}\neq r}(p^{\mathrm{RR}}_{r,r',t} \eta^{\mathrm{TR}}) 
  + p^{\mathrm{SGEN}}_{r,t}+p^{\mathrm{WGEN}}_{r,t}
  - \sum_{v\in \mathcal{V}_r}p^{\mathrm{VR}}_{v,r,t}
  + \sum_{v\in \mathcal{V}_r}p^{\mathrm{DR}}_{v,r,t}
\end{multlined}
 &    \lcol{\forall r\in \mathcal{R}, t\in \mathcal{T}}
 & \parbox{2cm}{\textrm{Demand must equal supply}}
 \label{eq01} \\
 & -p^{\mathrm{RRMAX}}_{r,r',t} \le p^{\mathrm{RR}}_{r,r',t}\le p^{\mathrm{RRMAX}}_{r,r',t}    
 &   \lcol{\forall r\in \mathcal{R}, r' \in \mathcal{R}\neq r, t\in\ \mathcal{T}}
 & \parbox{2cm}{\textrm{Inter-region power flow limits}}
 \label{eq2} \\
 & 0 \le p^{\mathrm{SGEN}}_{r,t}\le p^{\mathrm{SGENMAX}}_{r,t}    
 &   \lcol{\forall r\in \mathcal{R}, t\in\ \mathcal{T}}
 & \parbox{2cm}{\textrm{Regional solar PV output limits}}
 \label{eq21} \\
 & 0 \le p^{\mathrm{WGEN}}_{r,t}\le p^{\mathrm{WGENMAX}}_{r,t}    
 &   \lcol{\forall r\in \mathcal{R}, t\in\ \mathcal{T}}
 & \parbox{2cm}{\textrm{Regional wind generator output limits}}
 \label{eq21} \\
&\phi^{SPIN}*P_{r,t}= \sum_{i\in \mathcal{I}_r}a^{\mathrm{GENSP}}_{i,t}+ \sum_{k\in \mathcal{K}_r}a^{\mathrm{STRSP}}_{k,t} +\sum_{v\in \mathcal{V}_r}a^{\mathrm{VRSP}}_{v,r,t}
 &   \lcol{\forall r\in \mathcal{R}, t\in\ \mathcal{T}}
 & \parbox{2cm}{\textrm{Regional spinning reserve requirement must be met}}
 \label{eq:Rspinconstraint21} \\
 & \phi^{REG}*P_{r,t}+\phi^{\mathrm{WIND}}*p^{\mathrm{WGEN}}_{r,t}
 +\phi^{\mathrm{SOL}}*p^{\mathrm{S}}
 = \sum_{i\in \mathcal{I}_r}a^{\mathrm{GEN}}_{i,t}+ \sum_{k\in \mathcal{K}_r}a^{\mathrm{STR}}_{k,t} +\sum_{v\in \mathcal{V}_r}a^{\mathrm{VR}}_{v,r,t}
 &   \lcol{\forall r\in \mathcal{R}, t\in\ \mathcal{T}}
 & \parbox{2cm}{\textrm{Regional regulation reserve requirement must be met}}
 \label{eq:Rspinconstraint21} \\ 
 % storage constraints
\multicolumn{4}{l}{Storage constraints:} \nonumber \\
 & c^{\mathrm{STR}}_{k,t+1}=c^{\mathrm{STR}}_{k,t}+
 p^{\mathrm{STRC}}_{k,t} \eta^{STR}-p^{\mathrm{STRD}}_{k,t}/\eta^{STR}
 &    \lcol{\forall k\in \mathcal{K}, \forall t\in \mathcal{T}}  
 & \parbox{2cm}{\textrm{Storage state of charge}}
  \label{eq1str} \\
 & c^{\mathrm{STRMIN}}_{k} \le c^{\mathrm{STR}}_{k,t}\le c^{\mathrm{STRMAX}}_{k}  & \lcol{\forall k\in \mathcal{K}, \forall t\in \mathcal{T}}
 & \parbox{2cm}{\textrm{Storage unit capacity}}
 \label{eq2str} \\
  & 0\leq p^{\mathrm{STRD}}_{k,t},p^{\mathrm{STRC}}_{k,t}\leq p^{\mathrm{STRMAX}}_{k}  & \lcol{\forall k\in \mathcal{K}, \forall t\in \mathcal{T}}
 & \parbox{2cm}{\textrm{Max charge and discharge of storage units}}
 \label{eq3str} \\
   & a^{\mathrm{STRSP}}_{k,t}\leq p^{\mathrm{STRMAX}}_{k}
   - p^{\mathrm{STRD}}_{k,t}+p^{\mathrm{STRC}}_{k,t}-a^{\mathrm{STR}}_{k,t}
   & \lcol{\forall k\in \mathcal{K}, \forall t\in \mathcal{T}}
 & \parbox{2cm}{\textrm{Max regulation and spinning reserve offer of storage units}}
 \label{eq4str} \\
    & a^{\mathrm{STRSP}}_{k,t}\leq c^{\mathrm{STR}}_{k,t}-a^{\mathrm{STR}}_{k,t}
   & \lcol{\forall k\in \mathcal{K}, \forall t\in \mathcal{T}}
 & \parbox{2cm}{\textrm{Max regulation and spinning reserve offer of storage units}}
 \label{eq5str} \\
 %BEV constraint
\multicolumn{4}{l}{BEV constraints:} \nonumber \\
 & c^{\mathrm{VEH}}_{v,r,t+1}=c^{\mathrm{CHG}}_{v,r,t} -  p^{\mathrm{VR}}_{v,r,t}\eta^{\mathrm{VEH}}_v - N_{v,r}M_{v,t}\eta^{\mathrm{VEH}}_{v}
 &    \lcol{\forall v\in \mathcal{V}, \forall r\in \mathcal{R}, \forall t\in \mathcal{T}}
 & \parbox{2cm}{\textrm{BEV state of charge}}
  \label{eq3} \\
 & 10\% N_{v,r}B_{v} \leq c^{\mathrm{VEH}}_{v,r,t} \leq 90\% N_{v,r}B_{v}
 &    \lcol{\forall v\in \mathcal{V}, \forall r\in \mathcal{R}, \forall t\in \mathcal{T}}
 & \parbox{2cm}{\textrm{BEV battery capacity}}
  \label{eq3}\\
 & c^{\mathrm{VEH}}_{v,r,t} = 90\% N_{v,r}B_{v}
 &    \lcol{\forall v\in \mathcal{V}, \forall r\in \mathcal{R}, \forall t=T_v^{\mathrm{DEPART}}}
 & \parbox{2cm}{\textrm{BEVs fully charged when departing home}}
  \label{eq3}\\
 & c^{\mathrm{VEH}}_{v,r,t+1}=c^{\mathrm{CHG}}_{v,r,t} -  p^{\mathrm{VR}}_{v,r,t}/\eta^{\mathrm{CHG}}_v + p^{\mathrm{DR}}_{v,r,t}*\eta^{\mathrm{CHG}}_v N_{v,r}M_{v,t}\eta^{\mathrm{VEH}}_{v}
 &    \lcol{\forall v\in \mathcal{V}, \forall r\in \mathcal{R}, \forall t\in \mathcal{T}}
 & \parbox{2cm}{\textrm{BEV state of charge}}
  \label{eq3} \\
 &  0 \leq p^{\mathrm{VR}}_{v,r,t},p^{\mathrm{DR}}_{v,r,t} \leq N_{v,r}A_{v,t}R^{\mathrm{MAXCHG}}_v
 &    \lcol{\forall v\in \mathcal{V}, \forall r\in \mathcal{R}, \forall t\in \mathcal{T}}
 & \parbox{2cm}{\textrm{BEV charge rate limit}}
  \label{eq3} \\
 & 0 \leq c^{\mathrm{VEH}}_{v,r,t} \leq N_{v,r}B_{v}
 &    \lcol{\forall v\in \mathcal{V}, \forall r\in \mathcal{R}, \forall t\in \mathcal{T}}
 & \parbox{2cm}{\textrm{BEV battery capacity}}
  \label{eq3}\\
     & a^{\mathrm{VRSP}}_{v,r,t}\leq N_{v,r}A_{v,t}R^{\mathrm{MAXCHG}}_v
   - p^{\mathrm{DR}}_{v,r,t}+p^{\mathrm{VR}}_{v,r,t}-a^{\mathrm{VR}}_{v,r,t}
   & \lcol{\forall r\in \mathcal{R},\forall v\in \mathcal{V}}
 & \parbox{2cm}{\textrm{Max regulation and spinning reserve offer of vehicle fleets}}\nonumber
 \label{eq4str} \\
    & a^{\mathrm{VRSP}}_{v,r,t}\leq c^{\mathrm{VEH}}_{v,r,t}-a^{\mathrm{VR}}_{v,r,t}
   & \lcol{\forall r\in \mathcal{R},\forall v\in \mathcal{V}, \forall t\in \mathcal{T}}
 & \parbox{2cm}{\textrm{}}
 \label{eq5str} \\
\end{align}
\end{table}

\begin{table}
\centering
  \caption{Optimization problem formulation.}
  \label{tb:eqns}
   \scriptsize

\allowdisplaybreaks
\begin{align}
\multicolumn{4}{l}{
\begin{multlined} \textrm{Minimize } Z =
\sum_{t\in \mathcal{T}}{\sum_{i\in I_r} {\left({C}^{\mathrm{VARGEN}}_i  p^{\mathrm{GEN}}_{i,t} + s^{\mathrm{STARTUP}}_{i,t}+C^{\mathrm{GENRES}}_{i} a^{\mathrm{GEN}}_{i,t} \right)}}\\
+\sum_{t\in \mathcal{T}}{\sum_{k \in \mathcal{K}}{C^{\mathrm{STR}}_{k} a^{\mathrm{STR}}_{k,t}}}+\sum_{t\in \mathcal{T}}{\sum_{v \in \mathcal{V}}{\sum_{r \in \mathcal{R}}}{C^{\mathrm{VEH}} a^{\mathrm{VEH}}_{v,r,t}}} 
\end{multlined}
} \nonumber \\ \text{Subject to}\nonumber \\
 % Generator constraints
\multicolumn{4}{l}{Generator constraints:} \nonumber \\
 & a^{\mathrm{GEN}}_{i,t} \le 1/12*\left(p_{i,t-1}+R_i u_{i,t} 
 + p^{\mathrm{GENMIN}}_{i} (u_{i,t}-u_{i,t-1}) -p_{i,t} \right)
 -a^{\mathrm{GENSP}}_{i,t} 
 & \lcol{\forall i\in \mathcal{I}\neq 1,\forall t\in \mathcal{T}}
 & \parbox{2cm}{\textrm{Generator regulation reserve offer constraints}} \label{eq:1} \\
 & a^{\mathrm{GEN}}_{i,t} \le 1/12*\left(p^{\mathrm{GENMAX}}_{i} -p_{i,t} \right)
 -a^{\mathrm{GENSP}}_{i,t} 
 & \lcol{\forall i\in \mathcal{I}\neq 1,\forall t\in \mathcal{T}}
 & \parbox{2cm}{\textrm{Generator regulation reserve offer constraints}} \label{eq:2} \\
  & a^{\mathrm{GEN}}_{i,t} \le 1/12*\left(p^{\mathrm{GENMAX}}_{i} u_{i,t} \right)
 -a^{\mathrm{GENSP}}_{i,t} 
 & \lcol{\forall i\in \mathcal{I}\neq 1,\forall t\in \mathcal{T}}
 & \parbox{2cm}{\textrm{Generator regulation reserve offer constraints}} \label{eq8} \\
 & a^{\mathrm{GENSP}}_{i,t} \le 1/6*\left(p_{i,t-1}+R_i u_{i,t} 
 + p^{\mathrm{GENMIN}}_{i} (u_{i,t}-u_{i,t-1}) -p_{i,t} \right)
 -a^{\mathrm{GENSP}}_{i,t} 
 & \lcol{\forall i\in \mathcal{I}\neq 1,\forall t\in \mathcal{T}}
 & \parbox{2cm}{\textrm{Generator spinning reserve offer constraints}} \label{eq8} \\
 & a^{\mathrm{GENSP}}_{i,t} \le 1/6*\left(p^{\mathrm{GENMAX}}_{i} -p_{i,t} \right)
 -a^{\mathrm{GEN}}_{i,t} 
 & \lcol{\forall i\in \mathcal{I}\neq 1,\forall t\in \mathcal{T}}
 & \parbox{2cm}{\textrm{Generator spinning reserve offer constraints}} \label{eq8} \\
  & a^{\mathrm{GENSP}}_{i,t} \le 1/6*\left(p^{\mathrm{GENMAX}}_{i} u_{i,t} \right)
 -a^{\mathrm{GEN}}_{i,t} 
 & \lcol{\forall i\in \mathcal{I}\neq 1,\forall t\in \mathcal{T}}
 & \parbox{2cm}{\textrm{Generator spinning reserve offer constraints}} \label{eq8} \\ 
 & s^{\mathrm{STARTUP}}_{i,t} \geq \left(u_{i,t} - u_{i,t-1}\right)C^{\mathrm{STARTGEN}}_i
 & \lcol{\forall i\in \mathcal{I},\forall t\in \mathcal{T}}
 & \parbox{2cm}{\textrm{Slack variable reflects\\startup decision}} \label{eq8} \\
 & s^{\mathrm{STARTUP}}_{i,t} \geq 0
 & \lcol{\forall i\in \mathcal{I},\forall t\in \mathcal{T}}
 & \parbox{2cm}{\textrm{Slack variable is\\nonnegative}\label{eq8}}  \\
 & p^{\mathrm{GENMIN}}_{i}  u_{i,t}\le p_{i,t}  \le p^{\mathrm{GENMAX}}_{i}  u_{i,t}
 & \lcol{\forall i\in \mathcal{I},\forall t\in \mathcal{T}}
 & \parbox{2cm}{\textrm{Min and max output\\of online generators}} \label{eq8} \\
 &
\begin{multlined}
  p^{\mathrm{GEN}}_{i,t}\le p^{\mathrm{GEN}}_{i,t-1}+R_i  u_{i,t-1} + p^{\mathrm{GENMIN}}_{i} \left(u_{i,t}-u_{i,t-1}\right)
\end{multlined}
 & \lcol{\forall i\in \mathcal{I},\forall t\in \mathcal{T}}
 & \parbox{2cm}{\textrm{\raggedleft Ramp rate limit}} \label{eq9a} \\
 &
\begin{multlined}
  p^{\mathrm{GEN}}_{i,t-1}\leq p^{\mathrm{GEN}}_{i,t} + R_i  u_{i,t} + p^{\mathrm{GENMIN}}_{i} \left(u_{i,t-1} - u_{i,t}\right)
\end{multlined}
 & \lcol{\forall i\in \mathcal{I},\forall t\in \mathcal{T}}
 & \parbox{2cm}{\textrm{\raggedleft Ramp rate limit}} \label{eq9b} \\
 & \sum_{t=2}^{T_i^{\mathrm{UPMIN}}-T_i^{\mathrm{UPSTART}}}{\left(1-u_{i,t}\right)=0}
 & \lcol{\forall i\in \mathcal{I}}
 & \parbox{2cm}{\textrm{Minimum uptime\\(first timesteps)}} \label{eq10} \\
 & \sum_{t' =t}^{t+T_{i}^{\mathrm{UPMIN}}-1} u_{i,t'}\geq T_{i}^{\mathrm{UPMIN}}\left(u_{i,t}-u_{i,t-1}\right)
 &\lcol{\parbox{4.5cm}{
     \forall i\in \mathcal{I},\\ \forall t: T^{\mathrm{UPMIN}}_i-T^{\mathrm{UPSTART}}_i+1 \leq t  \leq |\mathcal{T}|-T_{i}^{\mathrm{UPMIN}}+1
    }}
 & \parbox{2cm}{\textrm{Minimum uptime\\(middle timesteps)}} \label{eq11} \\
 & \sum_{t=t}^{T}{u_{i,t'} \geq \left(T-t\right) \left(u_{i,t}-u_{i,t-1}\right)} &
 \lcol{
 \parbox{2cm}{\forall i \in \mathcal{I}, \\ \forall t\in |\mathcal{T}|-T_{i}^{\mathrm{UPMIN}}+2 \ldots }}
 & \parbox{2cm}{\textrm{Minimum uptime\\(final timesteps)}}\label{eq12} \\
 & \sum_{t=2}^{T^{\mathrm{DTMIN}}_{i}-T^{\mathrm{DTSTART}}_{i}}{u_{i,t}=0} & \lcol{\forall i\in\mathcal{I}}
 & \parbox{2cm}{\textrm{Minimum downtime\\(first timesteps)}}
 \label{eq13} \\
 & \sum_{t' =t}^{t+T^{\mathrm{DTMIN}}_{i}-1} 1-u_{i,t'}\geq T^{\mathrm{DTMIN}}_{i}\left(u_{i,t-1}-u_{i,t}\right)
 & \lcol{
 \parbox{2cm}{\forall i\in \mathcal{I},\\ \forall t: T^{\mathrm{DTSTART}}_{i}+1 \leq t \leq |\mathcal{T}|-T^{\mathrm{DTSTART}}_{i}+1}}
 & \parbox{2cm}{\textrm{Minimum downtime\\(middle timesteps)}} \label{eq14} \\
  & \sum_{t'=t}^{T}{1 - u_{i,t'} \geq \left(|\mathcal{T}|-t\right) \left(u_{i,t-1}-u_{i,t}\right)} &
  \lcol{
  \parbox{2cm}{\forall i \in \mathcal{I}, \\ \forall t: |\mathcal{T}|-T^{\mathrm{DTSTART}}_{i}+2 \leq t \leq |\mathcal{T}|}}
 & \parbox{2cm}{\textrm{Minimum downtime\\(final timesteps)}}\label{eq15}
\end{align}
\end{table}


This study models two reserve products: regulation reserve (5min) and spinning reserve (10min). For the calculation of these requirements, we adopt the parameters developed by the National Renewable Energy Laboratory (NREL) ReEDS model \cite{sergi_operating_2021}. The reserve provision costs account for variable costs incurred by degraded heat rate and other factors, and we adopt the parameters used by NREL's ReEDS model and Craig et al. 2017 (seen in Table \ref{tb:rescost}) \cite{sergi_operating_2021, craig_economic_2017}.

\begin{table}[!ht]
    \centering
    \label{tb:rescost}
    \caption{Reserve provisional costs of different reserve providers by fuel type or technology type}
    \begin{tabular}{llllll}
    \hline
        Fuel and technology type & Coal & Combined Cycle Natural Gas & Gas/Oil steam  & Storage & Storage by BEV \\ \hline
        Cost(\$/MWh) & 10 & 6 & 4 & 2 & 2 \\ \hline
    \end{tabular}
\end{table}

\section*{More results}

\newpage
\begin{figure}[!ht]
    \centering
    \includegraphics{Figures/emi&dmg_cpboth_5.pdf}
    \caption{Annual emissions and environmental damages of greenhouse gases and air pollutants, by different grid intervention scenarios and analysis approaches. VRE capacity varies across grid intervention scenarios in VRE scenarios, as PEV charging intervention intervention affects the maximum cost-effective VRE capacity installation. Environmental damages for GHG emissions are calculated with social cost of carbon of \$204/ton $CO_2e$. The AP3 model is used to calculate those of pollutants.}
    \label{fig:emidmg51}
\end{figure}
\newpage
\begin{figure}[!ht]
    \centering
    \includegraphics{emidmg_6_SI.pdf}
    \caption{Annual environmental damages of GHG emissions and pollution, by different grid intervention scenarios and analysis approaches. VRE capacity varies across grid intervention scenarios in Expansion scenarios, as grid intervention affects the maximum cost-effective VRE capacity installation. Environmental damages for GHG emissions are calculated with social cost of carbon of \$51/ton $CO_2e$. The AP3 model is used to calculate those of pollutants.}
    \label{fig:emidmg51}
\end{figure}
%%% Each figure should be on its own page


%%% Add this line AFTER all your figures and tables
\FloatBarrier


\bibliography{references}

\end{document}
